\chapter{Fundamente teoretice}\label{ch:3fundamenteTeoretice}

	În acest capitol voi prezenta principalele concepte teoretice utilizate în realizarea proiectului, împreună cu limbajele de programare și framework-urile folosite. 

\section{Python}

	"Python este un limbaj de programare popular. Acesta a fost creat în anul 1991 de către Guido van Rossum." \cite{python}. Se bucură de o evoluție fulminantă, ajungând să fie unul din cele mai utilizate limbaje de programare în anul 2020. Creșterea numărului de programatori care aleg să folosească Python, este datorată  caracteristicilor precum: 
	\begin{itemize}
	\setlength{\itemindent}{2em}
	\itemsep0em
	\item Flexibilitatea, poate fi utilizat într-un număr vast de domenii, de la programare web, până la programare pe plăcuțe. Funcționează pe o multitudine de platforme, printre care se enumeră: Windows, Mac, Linux și Raspberry Pi \cite{python}. 
	\item În ceea ce privește sintaxa acestui limbaj de programare, este una simplă, care permite scrierea de programe utilizând un număr mai mic de linii de cod \cite{python}. 
	\item Poate fi folosit atât pentru programare procedurală, funcțională, dar și orientată pe obiecte \cite{python}.
	\end{itemize}

\section{Flask}

	Este un framework construit pe baza limbajului de programare Python, ce oferă posibilitatea de a dezvolta aplicații web. Faptul că este proiectat ca să fie extins, oferă posibilitatea programatorului de a avea control total asupra aplicației pe care o creează. Prezintă un nucleu robust, care include toate funcționalitățile de bază pe care o aplicație web le necesită, nucleu ce poate fi extins de diverse părți terțe \cite{flask}.

	Pentru a crea aplicații web complexe, utilizarea doar a framework-ului nu este suficientă. Motiv pentru care, flask permite îmbinarea cu limbaje de programare precum javascript, CSS și HTML.

\section{C++}

	C++ este un limbaj de programare bazat pe C. Motivele pentru care creatorul acestui limbaj de programare, Bjarne Stroustrup, a decis să folosească limbajul C ca punct de plecare sunt: flexibilitatea si faptul că este un limbaj apropiat de partea hardware, rulează pe multe platforme și se potrivește cu mediul de programare UNIX \cite{c++}.

	Ceea ce aduce nou este posibilitatea de a programa orientat pe obiecte, o programare de tip generic și face posibil conceptul de abstractizare al datelor \cite{c++}. Numărul de domenii în care C++ poate fi aplicat crește considerabil datorită introducerii noțunii de programare orientată pe obiecte. Aceasta implică modelarea unor entități din lumea reală, și interacțiunile acestora, prin intermediul claselor. 

\section{Arduino IDE}

	

	Pentru programarea plăcuțelor Arduino, se folosește un mediu de dezvoltare numit Arduino Integrated Development Environment. Acesta suportă limbajele de programare C si C++. 

	Poate rula pe mai multe platforme precum: Windows, Linux, MAC și Java. De asemenea, este important de menționat faptul că este compatibil cu o serie largă de modele de plăcuțe, câteva exemple fiind:
		\begin{itemize}
			\setlength{\itemindent}{2em}
			\itemsep0em
			\item Arduino Uno
			\item Arduino Mega
			\item Arduino Leonardo
			\item Arduino Micro \cite{arduinoIDE}
		\end{itemize} 

	Arduino IDE îndeplinește atât rolul de editor de text, cât și rolul de compilator. Editorul de text reprezintă un suport pentru redactarea codului, iar compilatorul este responsabil de transformarea codului sursă în cod obiect și încărcarea acestuia pe microcontroler \cite{arduinoIDE}. Meniul este unul simplu, fapt ce face acest program ușor de utilizat

\section{Componente utilizate}
